\documentclass[oneside, final, 14pt]{article}
\usepackage[utf8]{inputenc}
\usepackage[russianb]{babel}
\usepackage{vmargin}
\usepackage{blindtext}
\setpapersize{A4}
\setmarginsrb{2cm}{1.5cm}{1cm}{1.5cm}{0pt}{0mm}{0pt}{13mm}
\usepackage{indentfirst}
\usepackage{graphicx}
\usepackage{amsmath}
\usepackage{amssymb}
\usepackage{amsthm}
\usepackage{algorithm2e}
\usepackage{mathrsfs}
\bibliographystyle{unsrt}

%\usepackage{biblatex}
\begin{document}

\pagenumbering{arabic}
\tableofcontents



\addtocontents{toc}{\protect\setcounter{tocdepth}{2}}


\newpage
\section*{Введение}
\addcontentsline{toc}{section}{Введение}
В документе приведено описание алгоритма восстановления волнового фронта, а также исследование зависимости точности восстановления от 
радиуса суперагаусса.

\newpage
\section{Описание алгоритма}
В вариационном методе задача восстановление формулируется в виде задачи минимизации целевого функционала невязки между производными 
искомой функции по $x,\,y$ и соответствующими наклонами волнового фронта $g_1,\, g_2$. Метод заключается в построении проекционно-разностной 
схемы и решении её методом Фурье. Рассмотрим функцию от двух переменных $u(x,y)$ , ее производные имеют вид $u_x(x,y)$, $u_y(x,y)$.
Введем сетку: 
\begin{equation}
\omega_h = \{(x_i, y_j) \,:\, x_i = ih_1,\,y_j = jh_2,\, i = 0..N_1,\, j = 0..N_2;\; x_{N_1} = 1, y_{N_2} = 1, x_0 = -1, y_0 = -1\}
\end{equation}
Матрицы $U,\, g_1,\, g_2$ являются дискретными представлениями функций $u(x,y) ,\, u_x(x,y) ,\, u_y(x,y)$ на сетке $\omega_h$. Задача восстановления волнового фронта состоит в приближенном восстановлении матрицы матрицы $U$, по известным наклонам $g_1,\, g_2$.

Рассмотрим функционал $$J(u) = \int \limits_{-1}^{1} \int \limits_{-1}^{1} ((u_x - g_1)^2 + (u_y-g_2)^2 + \alpha u^2 )dxdy$$
Функция $u$ принадлежит соболевскому пространству $W_{2}^{1}(\Omega)$, пространству периодических функций, определенных в области $\Omega = [-1,\,1]  \times [-1,\,1]$ и имеющих суммируемые с квадратом первые обобщенные производные по $x,\,y$. Минимизация функционала позволяет найти функцию $u$ с наименьшими квадратичными отклонениями градиента от измеренных локальных наклонах $g_1,\, g_2$. Дополнительное слагаемое $\alpha u^2$ , является регуляризатором Тихонова с параметром $\alpha$. Благодаря регуляризатору устраняются проблемы единственности минимума функционала и возможной погрешности исходных данных. Необходимым условием минимума функционала является:
\begin{equation}\label{nessasary condition}
(u_x, \phi_x) + (u_y,\phi_y) + \alpha(u, \phi) = (g_1, \phi_x) + (g_2, \phi_y),\;\;\forall\phi \in W_{2}^{1}(\Omega)
\end{equation}

Здесь $(\cdot\,,\cdot)$ - скалярное произведение в $L_2(\Omega)$.

Определим на $\Omega$ систему финитных базисных функций $\{\phi_i(x)\} ,\, i \in [0,\,N-1]$.
$$
\phi_i(x) = 
 \begin{cases}
  \frac{x-x_{i-1}}{h} , x \in [x_{i-1},x_i] \\ 
   \frac{x_{i+1} - x}{h}, x \in [x_{i},x_{i+1}]   \\
   0 \notin [x_{i-1},x_{i+1}].
 \end{cases} i \in [1, N-1];\;\;\;
\phi_0(x) = 
 \begin{cases}
  \frac{x_1-x}{h} , x \in [x_{0},x_1] \\ 
   \frac{x - x_{N-1}}{h}, x \in [x_{N-1},x_{N}]   \\
   0 \in [x_{1},x_{N-1}].
 \end{cases}
$$

Введем пространство конечных элементов $W_h = \mathscr{L}\{\phi_i(x)\phi_j(y)\}_{i,j=0}^{N_1 - 1,N_2-1} \in W_{2}^{1}(\Omega)$. 
Проекционная схема решения ($\ref{nessasary condition}$) состоит в нахождении функции $u^h \subset W_h$, которая при любых $k,\,l$ удовлетворяет тожеству:
\begin{equation}\label{mult diff scheme}
(u_x^h,\phi_x)+(u_y^h,\phi_y) + \alpha(u^h,\phi) = (g^1,\phi_x) + (g^2,\phi_y)
\end{equation}
Где:
$$\phi(x,y) = \phi_k(x)\phi_l(y)$$
$$u^h(x,y) = \sum \limits_{i=0}^{N_1-1} \sum \limits_{j=0}^{N_2-1} u_{ij}\phi_i(x)\phi_j(y)$$
$$u_{ij} = u(x_i,y_i)$$



Подставив в (\ref{mult diff scheme}) необходимые значения получим разностную схему
\begin{equation}
B_2 \Lambda_1 u + B_1 \Lambda_2 u + \alpha B_1 B_2 u = F(g_1,g_2)
\end{equation}
Правая часть $F(g_1,g_2)$ не определена однозначно и будет уникальной для каждого представления $g_1,\,g_2$.
Введем дополнительный регуляризатор $\gamma(\Lambda_1\Lambda_2)^s u$. В таком случае схема будет иметь вид:
\begin{equation}\label{scheme}
B_2 \Lambda_1 u + B_1 \Lambda_2 u + \alpha B_1 B_2 u +\gamma(\Lambda_1\Lambda_2)^s u = F(g_1,g_2)
\end{equation}
Разделим обе части равенства на $h_1h_2$. Это необходимо будет участь в дальнейшем при получении $F$.
В этом случае операторы $\Lambda_{1,2},\,B_{1,2}$ в матричной форме имеют вид:
$$\Lambda_i = \frac{1}{h_i^2}
\begin{bmatrix}
2 & -1 & 0 & \ldots & 0 & 0 & -1\\
-1 & 2 & -1 & \ldots & 0 & 0 & 0\\
0 & -1 & 2 & \ldots &0 & 0 & 0\\
\ldots & \ldots & \ldots & \ldots & \ldots & \ldots & \ldots &\\
0 & 0 & 0 & \ldots & -1 & 2 & -1\\
-1 & 0 & 0 & \ldots & 0 & -1 & 2
\end{bmatrix}
$$
$$
\qquad B_i = 
\begin{bmatrix}
\frac{2}{3} & \frac{1}{6} & 0 & \ldots & 0 & 0 & \frac{1}{6}\\
\frac{1}{6} & \frac{2}{3} & \frac{1}{6} & \ldots & 0 & 0 & 0\\
0 & \frac{1}{6} & \frac{2}{3} & \ldots &0 & 0 & 0\\
\ldots & \ldots & \ldots & \ldots & \ldots & \ldots & \ldots &\\
0 & 0 & 0 & \ldots & \frac{1}{6} & \frac{2}{3} & \frac{1}{6}\\
\frac{1}{6} & 0 & 0 & \ldots & 0 & \frac{1}{6} & \frac{2}{3} \\
\end{bmatrix} = E - \frac{h_i^2}{6}\Lambda_i$$
Благодаря делению на $h_1h_2\;\Lambda$ является оператором второй разностной производной.
Воспользуемся методом Фурье для решения разностной схемы. Будем искать собственные функции оператора $\Lambda$ в виде:
$$v_k = e^{ikx}$$
Напомним решение задачи Штурма-Лиувилля в $H_h$ для оператора второй разностной производной:
$$\begin{cases}
y_{\overline{x}x} + \lambda y_i = 0,\quad i = \overline{1\ldots N-1} \\
y_0 = y_N
\end{cases}$$
Подставив в уравнение получим собственные значения 
$$
\lambda_k = \frac{4}{h^2} \sin^2(\frac{kh}{2}),\quad k = \overline{0 \ldots N-1}
$$
Так как $B = E - \frac{h_i^2}{6}\Lambda_i$, то собственными значениями оператора B будут:
$$
\mu_k = 1 - \frac{h^2}{6}\lambda_k,\quad k = \overline{0 \ldots N-1}
$$

Будем искать решение в виде разложения по собственным функциям:
$$u_{ij} = \sum \limits_{k=0}^{N_1 - 1} \sum \limits_{l=0}^{N_2 - 1} \overline{u}_{kl}v_k(x_i)v_l(y_j)$$
Подставив в разностное уравнение наше решение, и используя линейную независимость собственных функций, получим:
$$\mu_l \lambda_k + \mu_k \lambda_l + \alpha \mu_k \mu_l + \gamma \lambda_k \lambda_l = f_{kl}$$
Где $f_{kl}$ - преобразование Фурье $F(g_1,g_2)$.
Таким образом:
$$u_{kl} = \frac{f_{kl}}{\mu_l \lambda_k + \mu_k \lambda_l + \alpha \mu_k \mu_l + \gamma (\lambda_k \lambda_l)^s}$$

Как было отмечено выше вид $g_1,\,g_2$ определен неоднозначно. Рассмотрим несколько вариантов представления $g_1, g_2$ и получим для каждого $F(g_1,g_2)$.

$g_{1,2}(x,y) = \sum \limits_{i=0}^{N_1-1} \sum \limits_{j=0}^{N_2-1}g_{1,2}^{ij}\phi_i(x)\phi_j(y),\, g_{ij} = g(x_i,y_j)$

Учитывая то, что правая часть в (\ref{scheme}) была нормирована на $h_1h_2$
\begin{equation}\label{F}
F = \frac{1}{h_1h_2}((g_1,\phi_x) + (g_2,\phi_y))
\end{equation} 
\subsubsection{Разложение точных значений производных по базисным функциям $W_h$}
$$g_{1,2}(x,y) = \sum \limits_{i=0}^{N_1-1} \sum \limits_{j=0}^{N_2-1}g_{1,2}^{ij}\phi_i(x)\phi_j(y),\, g_{ij} = g(x_i,y_j)$$
Найдем $F(g_1,g_2)$. 
Пусть $F_1 = \frac{1}{h_1h_2}(g_1,\phi_x),\, F_2 = \frac{1}{h_1h_2}(g_2, \phi_y)$.

\begin{align*}
&F_1 = (g_1, \phi_x) = \sum \limits_{n = 0}^{N_1 - 1} \sum \limits_{m = 0}^{N_2 - 1} \int \limits_0^{2\pi}\int \limits_0^{2\pi} g_1^{nm}\phi_n(x)\phi_m(y) \phi'_k(x) \phi_l(y) dx dy =\\& 
\sum \limits_{n = 0}^{N_1 - 1} \sum \limits_{m = 0}^{N_2 - 1} g_1^{nm} \int \limits_0^{2\pi} \phi_n(x) \phi'_k(x) dx \int \limits_0^{2\pi} \phi_m(y) \phi_l(y) dy
\end{align*}

$\int \limits_0^{2\pi} \phi_n(x) \phi'_k(x) dx  = G_{kn} =
\begin{cases} 
    0, \, k = n \\ 
    -1, \, k - n = -1 \\
    1, \, n - k = 1 \\
    0, else
\end{cases}
\int \limits_0^{2\pi} \phi_m(y) \phi_l(y) dy = B_{ml} =
\begin{cases}
    \frac{2}{3}, \, m = l \\
    \frac{1}{6}, \mid m-l \mid = 1\\
    0, else
\end{cases}$

$(g_1, \phi_x) = \sum \limits_{n = 0}^{N_1 - 1} \sum \limits_{m = 0}^{N_2 - 1} g_1^{nm} G_{kn} B_{ml} = 
\sum \limits_{n = 0}^{N_1 - 1} \big(\sum \limits_{m = 0}^{N_2 - 1} G_{kn} g_1^{nm} \big) B_{ml} = G_1 g_1 B_2
$

Аналогично получаем, что $F_2 = B_1 g_2 G_2$.
Таким образом $F = G_1 g_1 B_2 + B_1 g_2 G_2$.
Матрицы $G_1,\,G_2$ имеют следующий вид:
$$G_1 = \frac{1}{2h_1}
\begin{bmatrix}
0 & -1 & 0 & \ldots & 0 & 0 & 1\\
1 & 0 & 1 & \ldots & 0 & 0 & 0\\
0 & 1 & 0 & \ldots &0 & 0 & 0\\
\ldots & \ldots & \ldots & \ldots & \ldots & \ldots & \ldots &\\
0 & 0 & 0 & \ldots & 1 & 0 & -1\\
-1 & 0 & 0 & \ldots & 0 & 1 & 0
\end{bmatrix}
$$
$$G_2 = \frac{1}{2h_2}
\begin{bmatrix}
0 & 1 & 0 & \ldots & 0 & 0 & -1\\
-1 & 0 & 1 & \ldots & 0 & 0 & 0\\
0 & -1 & 0 & \ldots &0 & 0 & 0\\
\ldots & \ldots & \ldots & \ldots & \ldots & \ldots & \ldots &\\
0 & 0 & 0 & \ldots & -1 & 0 & 1\\
1 & 0 & 0 & \ldots & 0 & -1 & 0
\end{bmatrix}
$$

$$G_2 = \frac{h_1}{h_2} G_1^T$$


\newpage
\section{Исследование точности восстановления вариционного метода от радиуса супергаусса}
Необходимо исследовать точность восстановления супергаусса от радиуса суперагусса.


В качетсве области берется квадрат $\Omega = [-1,1] \times  [-1,1]$, используется прямоугольная сетка размером $512 \times 512$. 
В качестве исходного волнового фронта берутся полиномы Цернике, заданные на всей области $\Omega$. Обозначим исходный в.ф. как $U$. 
Наклонами волнового фронта считаем первые производные по $x, y$, вычисленные аналитически, умноженные на супергаусс. Умножение на супрегаусс обеспечивает периодичность наклонов.
Восстанавливаем волновой фронт.
Полученный в.ф. обозначим как $\widetilde{U}$. Сравниваются отнонормированные исходный волновой фронт $U$ и восстановленный $\widetilde{U}$ в круге($\Omega_0$), радиусом $r_c = 0.4 r$.

Параметры метода : $\alpha = 10^{-4}, \gamma = 0.0028, s=0.842$.

Формула супергаусса:
$$
e^{-(\frac{x^2 + y^2}{r^2})^N}
$$
где, $r$ - радиус супергаусса, $N$ - степень.

В расчетах используется $N = 5$.
Алгоритм нормировки принимает на вход двумерный массив, отнимает от всех его элементов минимальный, а затем делит на максимальный. В результате получается массив значения которого лежат в сегменте $[0,1]$.


\begin{enumerate} 
 	\item $U_{tmp} = U - \min{U}$
   	\item $U_{result} = \frac{U_{tmp}}{\max(U_{tmp})}$
\end{enumerate}


В качетсве метрик для сравнения исходного и восстановленного в.ф. используются относительные нормы $\mathbb{C},\ \mathbb{L}_2$, где $u$ - исходный волновой фронт.
$$\Vert u - v\Vert_{\mathbb{C}} = \frac{\max\limits_{(x,y) \in \Omega_0} |u(x,y) - v(x,y)|}{\max\limits_{(x,y) \in \Omega_0} |u(x,y)|}$$
$$\Vert u - v\Vert_{\mathbb{L}_2} = \frac{\Vert u - v \Vert_{\mathbb{L}_2(\Omega_0)}}{\Vert u \Vert_{\mathbb{L}_2(\Omega_0)}}$$
\begin{algorithm}[H]
	\KwData{$U$ - исходный волновой фронт, $g_1$ - наклон по x, $g_2$ - наклон по y}
	$g_{1r}$ = \{\} , $g_{2r}$ = \{\} //наклоны умноженные на супергаусс соответсвующего радиуса r
	
	$U_r$ = \{\} //восстановленные волновые фронты при соответсвующих радиусах r
	
	$L_r$, $C_r$ = \{\}, \{\} //относительные нормы между исходным и восставноленным волновыми фронтами
	 
	\For{$r$ in supergauss\_radiuses} {
		$g_{1r}[r]$, $g_{2r}[r]$ = $g_1 * supergauss_r$, $g_2 * supergauss_r$
		
		$U_r[r] = method(g_{1r}[r],g_{2r}[r], \alpha, \gamma, s)$
		
		//normalize - нормировка волнового фронта
		
		//mask - круг, в котором сравниваются волновые фронты

		$mask = get\_mask(r_c = 0.4r)$
		
		$diff = normalize(U_r[r] * mask) - normalize(U * mask)$
		
		$L_r[r] = \Vert diff \Vert_{\mathbb{L}_2}$
		
		$C_r[r] = \Vert diff \Vert_{\mathbb{C}}$
	}
	\KwResult{$L_r, C_r$}
	\caption{Алгоритм}
\end{algorithm}
\begin{center}
Пример восстановления волнового фронта для полинома $Z_2^2 = \rho^2 \cos(2 \varphi) = (x^2 + y^2)\cos(2 \arctg\frac{x}{y} )$:
\end{center}
$$
\frac{d}{dx} Z_2^2 = 2 \rho \frac{d\rho}{dx}  \cos(2 \varphi) - 2\rho^2 \sin(2 \varphi) \frac{d\varphi}{dx} 
$$
$$
\frac{d}{dy} Z_2^2 = 2 \rho \frac{d\rho}{dy} \rho \cos(2 \varphi) - 2\rho^2 \sin(2 \varphi) \frac{d\varphi}{dy} 
$$

$$\frac{d\rho}{dx}  = \frac{x}{\sqrt{x^2 + y^2}}$$
$$\frac{d\rho}{dy}  = \frac{y}{\sqrt{x^2 + y^2}}$$

$$\frac{d\varphi}{dx}  = \frac{1}{y (1 + (\frac{x}{y})^2)}$$
$$\frac{d\varphi}{dy}  = -\frac{x}{y^2 + (1 + (\frac{x}{y})^2)}$$



\begin{figure}
\center{\includegraphics[scale=0.5]{r^2cos(2phi).png}}
\caption{Полином $Z_2^2$}
\label{fig:image}
\end{figure}

Зависимость точности восстановления от радиуса супергаусса:
\begin{figure}
\begin{minipage}[h]{0.49\linewidth}
\center{\includegraphics[width=0.9\linewidth]{C_norm_r^2cos(2phi).png}}
\end{minipage}
\hfill
\begin{minipage}[h]{0.49\linewidth}
\center{\includegraphics[width=0.9\linewidth]{L_2_normr^2cos(2phi).png}}
\end{minipage}

\label{ris:image1}
\end{figure}

\begin{center}
Пример восстановления волнового фронта для полинома $Z_3^3 = \rho^3 \cos(3\varphi) = 
\sqrt{x^2 + y^2}(x^2 + y^2)\cos(3 \arctg\frac{x}{y} )$:
\end{center}
$$
\frac{d}{dx} Z_3^3 = 3 \rho^2 \frac{d\rho}{dx}  \cos(3 \varphi) - 3\rho^3 \sin(3 \varphi) \frac{d\varphi}{dx} 
$$
$$
\frac{d}{dy} Z_3^3 = 3 \rho^2 \frac{d\rho}{dy} \rho \cos(3 \varphi) - 3\rho^3 \sin(3 \varphi) \frac{d\varphi}{dy} 
$$
\begin{figure}[h]
	\center{\includegraphics[scale=0.5]{r^3cos(3phi).png}}
	\caption{Полином $Z_3^3$}
	\label{fig:image}
\end{figure}

Зависимость точности восстановления от радиуса супергаусса:
\begin{figure}
	\begin{minipage}[h]{0.49\linewidth}
		\center{\includegraphics[width=0.9\linewidth]{C_norm_r^3cos(3phi).png}}
	\end{minipage}
	\hfill
	\begin{minipage}[h]{0.49\linewidth}
		\center{\includegraphics[width=0.9\linewidth]{L_2_normr^3cos(3phi).png}}
	\end{minipage}
	
	\label{ris:image1}
\end{figure}

\begin{center}
Пример восстановления волнового фронта для полинома $Z_5^5 = \rho^5 \cos(5\varphi) = 
\sqrt{x^2 + y^2}(x^2 + y^2)^2\cos(5 \arctg\frac{x}{y} )$:
\end{center}
$$
\frac{d}{dx} Z_5^5 = 5 \rho^4 \frac{d\rho}{dx}  \cos(5 \varphi) - 5\rho^5 \sin(5 \varphi) \frac{d\varphi}{dx} 
$$
$$
\frac{d}{dy} Z_5^5 = 5 \rho^4 \frac{d\rho}{dy} \rho \cos(5 \varphi) - 5\rho^5 \sin(5 \varphi) \frac{d\varphi}{dy} 
$$
\begin{figure}[h]
	\center{\includegraphics[scale=0.5]{r^5cos(5phi).png}}
	\caption{Полином $Z_5^5$}
	\label{fig:image}
\end{figure}

Зависимость точности восстановления от радиуса супергаусса:
\begin{figure}
	\begin{minipage}[h]{0.49\linewidth}
		\center{\includegraphics[width=0.9\linewidth]{C_norm_r^5cos(5phi).png}}
	\end{minipage}
	\hfill
	\begin{minipage}[h]{0.49\linewidth}
		\center{\includegraphics[width=0.9\linewidth]{L_2_normr^3cos(3phi).png}}
	\end{minipage}
	
	\label{ris:image1}
\end{figure}

\end{document}
